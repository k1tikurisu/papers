\documentclass[11pt,dvipdfmx]{jreport}
\usepackage{wuse_thesis}
\usepackage{indentfirst}
\usepackage{url}	% \url{}コマンド用.URLを表示する際に便利
\usepackage{graphicx}  % ←graphicx.styを用いてEPSを取り込む場合有効にする
\usepackage{listings}
\usepackage{multirow}

\usepackage{color}

\renewcommand{\lstlistingname}{Program}
\newcommand{\todo}[1]{\colorbox{yellow}{{\bf TODO}:}{\color{red} {\textbf{[#1]}}}}
			% 他のパッケージ・スタイルを使う場合には適宜追加

%%%%%%%%%%%%%%%%%%%%%%%%%%%%%%%%%%%%%%%%%%%%%%%%%%%%%%%%%%%%%%%%%%%%%%%%

%%
%% 主に表紙を作成するための情報
%%

%%  タイトル(修論の場合は英語表記も指定)
\title{JavaScriptテストコード変更内容に基づく\\後方互換性損失の検出}
%\etitle{Test\\Test\\Test}

%%  著者名(修論の場合は英語表記も指定)
\author{前川 大樹}
%\eauthor{Akinori Ihara}

%% 卒業論文・修士論文(以下のどちらかを選択)
\bachelar	% 卒業論文(4年生用)
%\master  	% 修士論文(M2用)

%%  学科・クラスタ
\department{システム工}
%\department{デザイン情報}
%\department{デザイン科学}

%%  学生番号
\studentid{60256245}

%%  卒業年度
\gyear{2023}		% 提出年が2022年なら,2021年度

%%  論文提出日
\date{2024年2月13日}	% 修士の場合は月(2021年2月)までとし,英語表記も指定
%\edate{February 2021}	% 修士の場合,こちら(英語表記)も有効化

%%%%%%%%%%%%%%%%%%%%%%%%%%%%%%%%%%%%%%%%%%%%%%%%%%%%%%%%%%%%%%%%%%%%%%%%

\begin{document}

\maketitle

%%
%%  概要
%%
\begin{abstract}
ソフトウェア開発では,ライブラリと呼ばれる再利用可能なプログラムの利用により,開発者自身が同じ機能を再実装する必要がなくなり開発効率が向上する.ライブラリは機能追加やバグ修正により頻繁に更新されているため,利用者は適宜ライブラリの更新を適用する必要がある.利用者がライブラリの更新を自身のソフトウェアに適用する際,利用者のソフトウェアが期待通りに動作しなくなることがある.このような,更新後のライブラリが利用者のソフトウェアに影響を与えることを後方互換性の損失という.ライブラリ開発者は,利用者が安全にライブラリ更新を適用できるように,後方互換性の損失を利用者に正確に伝える必要があるが,後方互換性を正確に判定することは困難である.従来研究では,ライブラリの動作を検証するテストコードの変更有無に着目した後方互換性の判定手法を提案している.しかし,テストコードの変更内容は考慮されておらず,誤検出も多い.そこで本研究では,テストコードの変更内容を考慮した後方互換性の判定手法を提案し,その精度を検証する.その結果\todo{結果}

\end{abstract}

%%  目次
\tableofcontents

%%  図目次 (図目次をいれたければ以下のコメントをはずす)
%\listoffigures

%%  表目次 (表目次をいれたければ以下のコメントをはずす)
%\listoftables

\newpage
\pagenumbering{arabic}	% 以降のページ番号を算用数字に

%%%%%%%%%%%%%%%%%%%%%%%%%%%%%%%%%%%%%%%%%%%%%%%%%%%%%%%%%%%%%%%%%%%%%%%%

%%
%%  本文はここから
%%

\chapter{はじめに}
ソフトウェア開発では,ライブラリと呼ばれる再利用可能なプログラムの利用により,開発者自身が同じ機能を再実装する必要がなくなり開発効率が向上する.ライブラリは機能追加やバグ修正により頻繁に更新されており,利用者は適宜ライブラリのバージョン更新が必要である.利用者が安全にライブラリを更新するために,パッケージマネージャはセマンティックバージョニング\footnote{\url{https://semver.org}}を採用し,バージョン間の互換性の有無を管理している.セマンティックバージョニングでは,ライブラリ更新をメジャー,マイナー,パッチのレベルに分類し,マイナーとパッチの更新では後方互換性を保つ変更が求められる.しかし,バージョン名の付与は開発者が手動で行うため,後方互換性が損失しているにも関わらず,誤ってマイナーやパッチに分類されてしまうことがある.
この問題は,特にJavaScriptのような動的な言語において顕著であり,ライブラリと利用者のコードの不整合が実行時まで検出されないという課題がある.
この課題に対して,JavaScriptライブラリの後方互換性を判定する研究が行われている.
 
松田らは,後方互換性を損失するライブラリの更新は,プログラムの更新と合わせてテストコードも修正すると考え,テストコードの変更有無による後方互換性の判定手法を提案した.
\cite{matsuda}
しかし,テストコードの変更内容を考慮しておらず,テストの誤り修正や実行手順の修正など,ライブラリ変更とは無関係のテストコード変更についても後方互換性を損失したと誤検出する.

本研究では,テストコードの変更内容に着目し,後方互換性の損失に関係するテスト変更内容を明らかにし,テスト変更内容に基づく後方互換性の判定手法の精度を評価する.具体的には,2つのリサーチクエスチョン(RQ)に回答する.
 
\begin{itemize}
\item RQ1:後方互換性の損失に関係するテストコード変更とは何か?
\item RQ2:テストコード変更内容に基づく後方互換性の判定手法の有効性はどの程度か?
\end{itemize}

\chapter{後方互換性の損失}

\section{クライアントへの影響}
ライブラリの後方互換性の損失とは,ライブラリのバージョン更新によって,更新前のバージョンとの互換性が損なわれることを指す.クライアントソフトウェアは,更新前のライブラリ仕様を前提とするため,後方互換性を損失するライブラリのバージョン更新を適用する際,コードの不整合によるエラーが起きることがある.この問題に対して,ライブラリ開発者は後方互換性の損失を含むライブラリ更新をクライアントに正確に伝えることが求められる.

ライブラリ開発者がクライアントに後方互換性の損失を伝える手法として,セマンティックバージョニング\footnote{\url{https://semver.org}}がある.セマンティックバージョニングは,バージョン名を付与するための規則でバージョン名はそのソフトウェアの変更点や互換性に関する情報を提供する.バージョン名は,{\verb|MAJOR.MINOR.PATCH|}の形式をとる.後方互換性を損失する変更では{\verb|MAJOR|},後方互換性を保つ変更では{\verb|MINOR|}または{\verb|PATCH|}の値を増やすことで後方互換性の有無をクライアントに伝える.

ライブラリ更新をメジャー,マイナー,パッチのレベルに分類し,バージョン名によって後方互換性の有無を,後方互換性を損失する変更ではメジャーを,後方互換性を保つ変更ではマイナーバージョンもしくはパッチバージョンを更新する.しかし,バージョン名の付与はライブラリ開発者が手動で行うため,後方互換性が損失しているにも関わらず誤ってマイナーやパッチに分類されてしまうことがある.



\section{後方互換性の損失の原因}
クライアントソフトウェアの開発者が依存ライブラリのバージョン更新を取り込む際,クライアントは更新前のライブラリ仕様を前提とするため不整合によるエラーが起きることがある.この問題に対して,ライブラリ開発者は利用者に影響があるライブラリ変更を意図せずバージョン更新に含めないことが求められる.

ライブラリ開発者によって新しいライブラリバージョンがリリースされた際,更新前のライブラリ仕様を前提としたクライアントソフトウェアとの不整合が起きることがある.

更新前のライブラリ仕様を前提としたクライアントと,更新後の


ライブラリの後方互換性の損失とは,新しいバージョンのライブラリがリリースされた際に,以前のバージョンとの互換性が損なわれることを指す.後方互換性を損失する更新が入ったライブラリバージョンをクライアントソフトウェアが適用すると,クライアントソフトウェアが期待通りに動作しなくなることがあるため,ライブラリ開発者は意図せず後方互換性を損失する変更をバージョン更新に含めないことが求められる.

\section{関連研究}

\subsection{ライブラリ変更に基づく後方互換性の検出手法}

\subsection{テストケースに基づく後方互換性の検出手法}

\section{キーアイデア}

\chapter{RQ1:後方互換性の損失に伴うテストコード変更とは何か?}\label{rq1}

\section{概要}
本章では,テストコード変更内容を分類し,実際に後方互換性が損失する時どのようなテストコード変更が行われるかを明らかにする.テストコード変更内容を分類するために,まず,JavaScript言語におけるテストコードの構成要素を定義し,テストコード変更内容の細分化を行う.その後,目視によりバージョン間のテストコード変更内容を集計する.

\section{調査方法}
\subsection{JavaScript言語におけるテストコード構成要素}
本研究では,プログラムのテストの中でも単体テストのみを対象とする.単体テストでは,関数やクラスといったプログラムを構成する単位が開発者の想定通りに動作するかを検証する.単体テストでは,〜〜前置き.

Program~\ref{testSample}は,JavaScriptで記述されたテストコードの例を示す.1行目で{\verb|sum|}関数をインポートしている.{\verb|sum|}関数は,2つの引数を受け取り,それらを足し合わせた結果を返す関数とする.3から10行目の{\verb|describe|}関数により,テストスイートとしてテストケースをまとめている.

JavaScript言語では,テストコードを書く際にフレームワークを使用することが多いが,フレームワーク毎に記法は異なるものの構成要素は変わらない.

\begin{figure}[t]
  \begin{lstlisting}[caption={[upper/lower text]%
             \begin{tabular}[t]{@{}l@{}}
              test/sum.test.js \\[1.0\normalbaselineskip]
             \end{tabular}},frame={tb},numbers=left,label=testSample,identifierstyle={\small}]
const sum = require('./sum');

describe('sum', function() {
  test('1+2=3', function() {
    expect(sum(1, 2)).toBe(3);
  });
  test('2-1=1', function() {
    expect(sum(2, -1)).toBe(1);
  });
});
\end{lstlisting}
\vspace{-6mm}
\end{figure}
\subsection{テストコード変更の細分化}


\section{調査結果}


\section{考察}

\chapter{RQ2:テストコード変更内容に基づく後方互換性の判定手法の有効性はどの程度か}

\section{概要}
本章では,\ref{rq1}章で述べた,後方互換性の損失に関係するテストコード変更を自動検出するツールを開発し,後方互換性の損失の検出精度を検証する.まず,抽象構文木を用いてテストコードを定義し,


% \subsubsection{テストコード内容の定義}
% JavaScriptの単体テストフレームワークとして使用される,JestやMochaの慣習に倣い,テストコード内容の定義を行う.

% \noindent\textbf{テストコードの定義:}関数呼び出しで,関数名がdescribeまたはitまたはtestであり,第一引数が文字列,第二引数が関数呼び出しのものを,テストコードとする.

% \noindent\textbf{アサーションの定義:}アサーションの記述形式はフレームワークにより異なるが,本分析では以下のように定義を行った.



\section{分析手法}

\subsection{テストコードの定義}

\subsection{変更パターンの作成}
JavaScript言語では,テストコードを書く際にフレームワークを使うことが多い.本調査では,State of JavaScript 2022\footnote{\url{https://2022.stateofjs.com/}}で紹介されている主要なテストフレームワーク13件のうち,単体テストで使われるフレームワーク5件を考慮する.

\subsection{変更パターンの検出}

\section{分析結果}

\section{考察}

\chapter{妥当性への脅威}

\section{内的妥当性}

\section{外的妥当性}

\chapter{おわりに}

\chapter*{謝辞}

文献を参照する場合には,論文の最後に参考文献として列挙するとともに,
\verb|\cite|を使って,例えば,
\begin{quote}
  文献\cite{1390850475731067264}によれば…
\end{quote}
や,
\begin{quote}
  …である\cite{latex2e}.
\end{quote}
のように参照する.

文献の列挙には,{\tt thebibliography}環境などを用いる\footnote{使い方
は,この資料のソースを参照.}.

%%%%%%%%%%%%%%%%%%%%%%%%%%%%%%%%%%%%%%%%%%%%%%%%%%%%%%%%%%%%%%%%%%%%%%%%

%%
%% 謝辞
%%
%% \begin{acknowledgements}
%% 感謝します.
%% \end{acknowledgements}

%%%%%%%%%%%%%%%%%%%%%%%%%%%%%%%%%%%%%%%%%%%%%%%%%%%%%%%%%%%%%%%%%%%%%%%%

%%
%% 参考文献
%%

\bibliographystyle{junsrt}
\bibliography{thesis}

%%%%%%%%%%%%%%%%%%%%%%%%%%%%%%%%%%%%%%%%%%%%%%%%%%%%%%%%%%%%%%%%%%%%%%%%

%%
%% 付録
%%
% \appendix
% 
% \chapter{サンプルプログラム}
% 
% プログラムリストや実行結果など,本論を補足する上で必要と思われるものが
% あれば付録として付ける.
% 
% {
% \footnotesize
% \begin{verbatim}
% #include <stdio.h>
% int main(void)
% {
%     printf("Hello, World!\n");
%     return 0;
% }
% \end{verbatim}
% }

%%%%%%%%%%%%%%%%%%%%%%%%%%%%%%%%%%%%%%%%%%%%%%%%%%%%%%%%%%%%%%%%%%%%%%%%

\end{document}
