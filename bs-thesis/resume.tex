%
% 卒論レジュメフォーマット Ver.2.0 pLaTeX版
%
\documentclass[twocolumn]{jarticle} % 2段組のスタイルを用いている

\usepackage{wuse_resume}
\usepackage{url}	% \url{}コマンド用.URLを表示する際に便利
\usepackage[dvipdfmx]{graphicx}  % ←graphicx.styを用いてEPSを取り込む場合有効にする
\usepackage{multirow}
			% 他のパッケージ・スタイルを使う場合には適宜追加

%%%%%%%%%%%%%%%%%%%%%%%%%%%%%%%%%%%%%%%%%%%%%%%%%%%%%%%%%%%%%%%%%%%%%%%%

%%
%% タイトル,学生番号,氏名などを設定する
%%

\タイトル{JavaScriptライブラリのテストコード変更内容に基づく\\後方互換性損失の検出}
\研究室{ソーシャルソフトウェア工学}
\学生番号{60256245}
\氏名{前川 大樹}

\概要{%
本研究では,ソフトウェアの後方互換性の損失をテストコードの変更内容に基づいて判定する手法を提案する.
ソフトウェア開発では,ライブラリと呼ばれる再利用可能なプログラムの利用により,開発者自身が同じ機能を再実装する必要がなくなり開発効率が向上する.ライブラリに対して行われる変更は,軽微な修正であっても破壊的変更が含まれることがあり,変更後のライブラリが後方互換性を維持しているか否かをライブラリ利用者が正確に判断することは困難である.
従来研究では,ライブラリの動作を検証するテストコードの変更有無に着目した後方互換性の損失の判定手法を提案している.しかし,テストコードはテストの誤り修正や実行手順の変更など,ライブラリの変更とは無関係に変更されることがあり,従来手法ではテストコードの変更内容は考慮されておらず誤検出が多い.
本研究では,従来研究の誤検出を減らすために,テストコードの変更内容を考慮した後方互換性の判定手法を提案する.具体的には,後方互換性が損失するライブラリバージョンにおけるテストコードの変更内容を明らかにし,自動検出するツールを開発して後方互換性の損失の判定精度を検証する.
}

\キーワード{ライブラリ}
\キーワード{後方互換性}
\キーワード{単体テスト}
\キーワード{JavaScript}
\キーワード{プログラム解析}

%%%%%%%%%%%%%%%%%%%%%%%%%%%%%%%%%%%%%%%%%%%%%%%%%%%%%%%%%%%%%%%%%%%%%%%%

%% 以下の3行は変更しない

\begin{document}
\maketitle
\thispagestyle{empty} % タイトルを出力したページにもページ番号を付けない

%%%%%%%%%%%%%%%%%%%%%%%%%%%%%%%%%%%%%%%%%%%%%%%%%%%%%%%%%%%%%%%%%%%%%%%%

%%
%% 本文 - ここから
%%

\section{はじめに}

ソフトウェア開発では,ライブラリと呼ばれる再利用可能なプログラムの利用により,開発者自身が同じ機能を再実装する必要がなくなり開発効率が向上する.ライブラリは頻繁に更新されており,ライブラリの利用者であるクライアントソフトウェア(以降,クライアント)は適宜ライブラリのバージョン更新が必要となる.クライアント開発者が依存ライブラリを更新する際,更新前の仕様を前提とするクライアントと,更新後ライブラリのソースコードに不整合が生じ,クライアントが実行時エラーになってしまうことがある.このような,更新後のライブラリがクライアントに影響を与えることを後方互換性を損失するという.ライブラリ開発者はクライアントが安全にライブラリを更新するために,バージョン名によって後方互換性の損失有無を利用者に共有するが,マイナー変更にもかかわらず後方互換性が損失していることも少なくない.JavaScriptのような動的な言語において,この問題は自身のソフトウェアを実行するまで検出されないことが多く,JavaScriptライブラリの後方互換性の損失を検出するための研究が行われている.

松田らは,後方互換性を損失するライブラリの更新は,プログラムの更新と合わせてテストコードも修正すると考え,テストの変更有無による後方互換性の判定手法を提案した\cite{matsuda}.しかし,テスト変更内容を考慮していないため,誤検出も多い.

本研究では,テストコードの変更内容を詳細に分析することで,後方互換性の損失をより正確に検出することを目指す.具体的には,2つのResearch Question(RQ)に回答する.

\begin{itemize}
  \item RQ1:後方互換性の損失に伴うテストコード変更とは何か?
  \item RQ2:テストコード変更内容に基づく後方互換性損失の検出手法の有効性はどの程度か?
\end{itemize}

RQ1では,ライブラリ更新が後方互換性の損失を含む際,それに伴ってテストコードをどのように変更するかを分析し,後方互換性の損失を検出する手がかりとなるテストコード変更内容を明らかにする.RQ2では,これらの変更内容を自動検出するツールを開発し,後方互換性損失の検出精度を検証する.

\section{RQ1:後方互換性の損失に伴うテストコード変更とは何か?}\label{rq1}

\subsection{調査方法}\label{rq1.chousahouhou}
本分析では,従来研究\cite{matsuda}のデータセットに含まれるライブラリバージョン2,111件から,ライブラリテストに変更がある1,027件を抽出し,95%の信頼区間でサンプリングした280件を目視で確認した.

\subsection{調査結果}

% ========================
\begin{table}[t]
\centering
\caption{テスト変更内容による後方互換性の損失有無}\label{fig:test_pattern}
\scalebox{0.95}[0.95]{
\begin{tabular}{l|r|r}
\hline
\multirow{2}{*}{テスト変更内容} & \multicolumn{2}{c}{後方互換性損失}  \\ \cline{2-3}
 & \multicolumn{1}{c|}{有り} & \multicolumn{1}{c}{無し} \\ \hline
テスト追加 & 104(76\%) & 32(24\%) \\ \hline
テスト削除 & 12(41\%) & 17(59\%) \\ \hline
期待値の変更 & 23(72\%) & 9(28\%) \\ \hline
リファクタリング & 129(83\%) & 27(17\%) \\ \hline
\end{tabular}
}
\end{table}

% ======================
\begin{table*}[t]
\centering
\caption{提案手法と従来手法の比較結果}
\label{fig:result}
\scalebox{0.75}{\begin{tabular}{cl|r|r|r}
\hline
\multicolumn{2}{c|}{} & \multicolumn{1}{c|}{後方互換性なし} & \multicolumn{1}{c|}{後方互換性あり} & \multicolumn{1}{c}{合計} \\ \hline
\multicolumn{1}{c|}{\multirow{3}{*}{提\newline 案\newline 手\newline 法}} & 後方互換性なしと判定      & 114 & 548 & 662 \\ \cline{2-5} 
\multicolumn{1}{c|}{}                                                      & 後方互換性ありと判定      & 109 & 1,184 & 1,293 \\ \cline{2-5} 
\multicolumn{1}{c|}{}                                                      & 合計              & 223 & 1,732 & 1,955 \\ \hline
\multicolumn{1}{c|}{\multirow{3}{*}{従\newline 来\newline 手\newline 法}} & 従来手法で後方互換性なしと判定 & 140 & 765 & 905 \\ \cline{2-5} 
\multicolumn{1}{c|}{}                                                      & 従来手法で後方互換性ありと判定 & 83  & 967 & 1,050 \\ \cline{2-5} 
\multicolumn{1}{c|}{}                                                      & 合計              & 223 & 1,732 & 1,955 \\ \hline
\end{tabular}}
\end{table*}
% ======================

% ========================

表\ref{fig:test_pattern}は,テスト変更内容による後方互換性損失の有無を示す.本論文では,紙面の都合上,特徴的な変更内容のみを示す.

ライブラリのテストが削除された場合,従来研究の仮説と同様に後方互換性を損失していた割合が高い.一方で,リファクタリングは件数も多くほとんどが実際には後方互換性を保っているため,従来研究の手法では誤検出となる.また,「テストスイート内でのテストコード追加」「テストコード削除」「入力値と期待値いずれか一方の変更」の3つが後方互換性の損失を検出する手掛かりになる,機械的に検出可能なテストコード変更であるとわかった.


\section{RQ2:テストコード変更内容に基づく後方互換性損失の検出手法の有効性はどの程度か?}



\subsection{提案手法}\label{teiannshuhou}
本研究では,\ref{rq1}章で述べた変更をそれぞれを検出するツールを開発し,後方互換性損失の検出精度を検証する.本ツールは,テスト内容を抽象構文木で定義し,コード差分解析ツールGumTree\cite{gumtree}を用いて変更情報を取得し,変更内容を次の方法で検出する.

\begin{description}
  \item[テストスイート内でのテストコード追加]GumTreeで「挿入」と判定された変更箇所がテストコードまたはアサーションであり,かつ既存のテストコード内で追加されていること
  \item[テストコード削除]GumTreeで「削除」と判定された変更箇所がテストコードであること
  \item[入力値と期待値いずれか一方の変更]GumTreeで「変更」と判定された変更箇所が入力値であり,同一アサーションで期待値がGumTreeで「変更」と判定されていないこと.またはその逆.
\end{description}

データセットは,\ref{rq1.chousahouhou}節と同様の従来研究\cite{matsuda}で収集されたライブラリバージョン2,111件を使用し,削除や非公開になったことによりGitHub上でアクセスできないものと,GumTree上でエラーになるもの計156件を除いた1,955件を使用する.

\subsection{分析結果}

データセットに対して,従来手法と\ref{teiannshuhou}節で述べた手法を適用した結果を表\ref{fig:result}に示す.提案手法の適合率は17\%,再現率は51\%であり,予測精度は低い結果となった.一方で,従来研究の問題点であった,リファクタリングなども後方互換性なしと誤検出してしまう課題は改善された.

\subsection{考察}
従来手法では,テストコード変更内容を考慮しないため後方互換性を損失したと判定するが,提案手法ではテストコード変更内容を考慮するため,後方互換性の損失とテストコード変更内容が無関係である場合検出することはできない.従来手法の検出精度は,ライブラリ本体の変更とテストコードの変更の関連性を十分に評価していないことに起因し,提案手法の検出精度の低下は,後方互換性を損失したと判定する条件を絞り込んだ結果と解釈できる.

また,後方互換性の有無のデータにはクライアントのテストコードを利用している.ライブラリの後方互換性が損失していても,影響を受けるクライアントが存在しない場合,後方互換性の損失を確認することができず,分析の精度を低下させる原因になる.今後の研究では正解データをより正確に収集し分析することが必要となる.

\section{おわりに}

本研究では,ライブラリの動作を検証するテストコードを利用した後方互換性の損失の検出手法を提案した.今後は,コールグラフを利用したテストコード追跡により,変数の中身やテストに影響を与えるテストコード外の変更を判定条件に含め,精度を向上させることが考えられる.

%%
%% 本文 - ここまで
%%

%%%%%%%%%%%%%%%%%%%%%%%%%%%%%%%%%%%%%%%%%%%%%%%%%%%%%%%%%%%%%%%%%%%%%%%%

%%
%% 参考文献
%%

\bibliographystyle{junsrt}
\bibliography{thesis}

%%%%%%%%%%%%%%%%%%%%%%%%%%%%%%%%%%%%%%%%%%%%%%%%%%%%%%%%%%%%%%%%%%%%%%%%

\end{document}
